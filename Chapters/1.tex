\chapterimage{intro.jpg} % Chapter heading image
\chapter{Introduzione}\label{1:chintro}

La Cosmologia è quel ramo della scienza che studia l'universo nel suo
insieme, per comprenderne l'origine e l'evoluzione, mediante strumenti
osservativi, fisici e matematici. In particolare la
cosmologia \emph{planckiana} è si occupa dell'universo nella sua
evoluzione a partire dal tempo di Planck (10\textsuperscript{-43} s). Si
parla di~\emph{evoluzione}~dell'universo da poco più di 100 anni, quindi
è una ``scienza'' giovane. I recenti risultati di Plank offrono le
seguenti stime:~\(\Omega_{M}=0.3, \Omega_{\Lambda}=0.7\) che qui verranno sottintese se non
diversamente indicato. Per quanto riguarda la materia
barionica,~\(\Omega_b=0.05\), ossia il 95\% dell'universo è a noi
sconosciuto. Il Modello Cosmologico Standard concorda in buona
parte con le osservazioni: quindi~\emph{descrive~}l'universo, ma non è
in grado~\emph{spiegare~}la fisica che ci sta sotto.

Nella prima parte del corso l'universo verrà trattato come un
contenitore di fluido, nella seconda parte verrà trattata la formazione
delle strutture attraverso piccole perturbazioni e instabilità
gravitazionali. I principi del Modello Cosmologico Standard sono due:

\begin{theorem}[Principio Cosmologico]
L'universo (almeno su grande scala) è~\emph{omogeneo} e~\emph{isotropo}.
Oggi per ``grande scala'' si intende circa 100 Mpc, dove l'isotropia è
osservata attraverso la distribuzione delle galassie ad alto $z$, ma il
concetto di ``grande'' dipende dal tempo cosmico.\label{th:princ1}
\end{theorem}

\begin{theorem}[La Gravità è ben descritta dalla Teoria della Relatività
Generale]
Le proprietà geometriche dello spazio sono legate alla quantità di
energia/materia ivi contenuta. Dal principio precedente ne consegue che
la distribuzione della materia è perfettamente omogenea e isotropa.\label{th:princ2}
\end{theorem}
In passato era in voga il principio cosmologico~\emph{perfetto:}~così
come non esistono direzioni privilegiate, non esistono tempi
privilegiati (e.g. Modello dello Stato Stazionario, 1948), ma è stato abbandonato in seguito alla scoperta della
radiazione cosmica di fondo.

\section{Metrica}\label{1:sec1metrica}

È necessaria una metrica per determinare una distanza tra due punti nello spazio-tempo. La definizione più generale affinché omogeneità e isotropia vengano preservate è:
\begin{equation}    
    \mathrm{d}s^2 = g_{\alpha \beta}\, x^\alpha x^\beta
  = g_{00}\mathrm{d}^2t+2g_{0i}\, \mathrm{d}t\mathrm{d}x^i+g_{ij}\mathrm{d}x^i\mathrm{d}x^j \qquad i,j=1,2,3
\end{equation}
dove~\(\alpha,\ \beta=0,\ 1,\ 2,\ 3\)~ e 0 è l'indice della componente temporale del
quadrivettore. Se il valore~\(\mathrm{d}s^2  = 0\) si parla di intervallo di
tipo~\emph{luce~} (i fotoni viaggiano su geodetiche nulle), se è
\textgreater{} 0 di tipo~\emph{tempo}, se è \textless{} 0 di
tipo~\emph{spazio}. Applicando i principi del Modello Standard:
\begin{equation}    
    \mathrm{d}s^2 = c \, \mathrm{d}t^2+g_{ij}\mathrm{d}x^i\mathrm{d}x^j
\end{equation}
ossia~\(g_{0i}=0\) affinché non ci sia una direzione privilegiata
nel tempo e~\(g_{00}=c^2\) affinché venga riprodotta
la traiettoria di un fotone. Il secondo termine, puramente spaziale, può essere scritto come: $-\d{l^2}$ (si assume una segnatura della matrice associata al tensore metrico: $+---$).

\par\null

\subsection{Superfici 2D}
Esistono 3 tipi di superfici bidimensionali in base alla curvatura $C$, parametrizzabili attraverso: raggio $\rho\in [0,+\infty)$, angolo polare $\theta\in [0,\pi)$ e azimuth $\varphi\in [0, 2\pi)$:

\vspace{1em}
\noindent\begin{tabular}{l |  l | l | l}
Geometria & $C$ & \\
\hline 
Euclidea & $0$  & $\mathrm{d}l^2=\mathrm{d}\rho^2+\rho^2\mathrm{d}\varphi^2$ & $\rho = ar$ \\
Sferica & $>0$ & $\mathrm{d}l^2=R^2(\sin^2\theta\; \mathrm{d}\varphi^2+\mathrm{d}\theta^2)$ & $R=a, \sin\theta =r $  \\
Iperbolica & $<0$ & $\mathrm{d}l^2=R^2(\sinh^2\theta\; \mathrm{d}\varphi^2+\mathrm{d}\theta^2) $ & $R=a, \sinh\theta =r $ \\
\end{tabular}

\vspace{1em}
Parametrizzandole in coordinate cilindriche è possibile trovare una
forma generale:
\begin{equation}
\mathrm{d}l^2=a^2\left ( r^2\mathrm{d}\varphi^2 + \frac{\mathrm{d}r^2}{1-kr^2}\right); \qquad k=\left\{\begin{matrix}
0 & \textrm{Euclidea} \\ 
+1 & \textrm{Sferica}  \\
-1 & \textrm{Iperbolica}
\end{matrix}\right.
\end{equation}

\subsection{Superfici 3D}
Analogamente, ponendo~\(\mathrm{d}\Omega^2=\mathrm{d}\theta^2+\sin^2\theta\, \mathrm{d}\varphi^2\), si ricavano le stesse
relazioni per le~ superfici tridimensionali:

\vspace{1em}
\noindent\begin{tabular}{l | l | l | l}
Geometria & $C$ & \\
\hline
Euclidea & $0$ & $\mathrm{d}l^2=R^2(\mathrm{d}r^2+r^2\mathrm{d}\Omega^2)$ & $R=a$ \\
Sferica & $>0$ & $\mathrm{d}l^2=R^2(\mathrm{d}\chi^2+\sin^2\chi\;\mathrm{d}\Omega^2)$ & $R=a, \sin\chi =r $  \\
Iperbolica & $<0$ & $\mathrm{d}l^2=R^2(\mathrm{d}\chi^2+\sinh^2\chi\;\mathrm{d}\Omega^2)$ & $R=a, \sinh\chi =r $ \\
\end{tabular}

\vspace{1em}
Riconducibili alla seguente metrica, detta di~\textbf{Robertson-Walker}
(\textbf{universo omogeneo, isotropo e velocità finita \emph{c} dei
fotoni}):
\begin{equation}
\mathrm{d}s^2=c^2\mathrm{d}t^2-a^2\left ( \frac{\mathrm{d}r^2}{1-kr^2} + r^2(\sin^2\theta\;\mathrm{d}\varphi^2+\mathrm{d}\theta^2) \right); \qquad k=\left\{\begin{matrix}
0 & \textrm{Euclidea} \\ 
+1 & \textrm{Sferica}  \\
-1 & \textrm{Iperbolica}
\end{matrix}\right.\label{eq1:frw}
\end{equation}

In precedenza sono state fatte alcune sostituzioni strategiche, si
definiscono quindi:

\begin{itemize}
\item
  \(a\)~~~~\textbf{\emph{fattore di scala (o di
  espansione)}}, che ha dimensioni di una lunghezza (la dimensionalità è
  contenuta esclusivamente in questo parametro) e può dipendete dal
  tempo (\(a=a\left(t\right)\))
\item
  \(k\)~ ~~\textbf{\emph{parametro di curvatura}}
\end{itemize}


\section{Distanze}\label{1:sec2distanze}

Esistono diverse definizioni possibili di `distanza'.

\subsection{Distanza propria}
Si assume di fare una misura istantanea (\(\mathrm{d}t=0\)) della
distanza tra sorgente e osservatore (non è possibile in relatività).
Inoltre si può assumere~ \(\mathrm{d}\theta=\mathrm{d}\varphi=0\) senza perdere di generalità:
\begin{equation}
    d_{PR}:=a\int_{0}^{r}\frac{\,\mathrm{d}r'}{\sqrt{1-kr^{'2}}}=a\; \mathfrak{f}(r) \qquad \mathfrak{f}(r):=\left\{\begin{matrix}
\arcsin r& k=+1\\ 
r & k=0\\ 
\mathrm{arcsinh}\:  r & k=-1
\end{matrix}\right.\label{eq1:mathfrankf}
\end{equation}

Se~\(a=a\left(t\right)\), allora~\(d_{PR}=d_{PR}(t)\) e si può derivare la
velocità radiale:
\begin{align*}
v_R & =\frac{\mathrm{d}d_{PR}}{\mathrm{d}t}=\frac{\dt{a}}{a}d_{PR}=H\,d_{PR}  \\
H & :=\frac{\dt{a}}{a}\quad [\mathrm{s^{-1}}]
\end{align*}

In base a questa relazione, detta~\textbf{legge di Hubble-Lemaitre},
esiste una velocità di recessione nella direzione radiale che è
proporzionale alla distanza. Il~\textbf{parametro di
Hubble}~\(H\), che può dipendere dal tempo, quantifica
questa proporzionalità. Si parla~ambiguamente di~\emph{costante di
Hubble~}riferendosi al fatto che,~\emph{oggi} in~\emph{ogni punto}
dell'universo, possiede lo stesso valore~\(H_0\). Facendone
l'inverso si ottiene il tempo di Hubble, ossia l'età che l'universo
avrebbe se si fosse espanso linearmente.

Si può verificare che la legge di Hubble è già implicita nel principio
cosmologico, ma questo non ci compete.


\begin{definition}[Wong et al., 2019]
Combines time delay data from 6 
quasars to get~\textbf{\textit{H}}\textsubscript{\textbf{0}} \textbf{= 73.3 $\pm$
1.75{~}km/sec/Mpc}, in agreement with the SH0ES Cepheid-based distance
ladder measurement (\textbf{74.03 $\pm$ 1.42}) and disagreeing with the
latest CMB model based value, \textbf{\textit{H}}\textsubscript{\textbf{0}}
\textbf{= 67.80 $\pm$ 0.9 km/sec/Mpc}. On the other hand Freedman et al. get
a distance ladder value of \textbf{\textit{H}}\textsubscript{\textbf{0}}
\textbf{= 69.8 $\pm$ 0.8 (stat) $\pm$ 1.7 (syst) km/sec/Mpc} using the tip of
the red giant branch to calibrate Type Ia supernovae.
\vspace*{0.5em}

From: Ned Wright's Website
\end{definition}

La \emph{distanza propria} calcolata ad \emph{oggi} è definita
\textbf{distanza comovente (o comobile)}:
t\begin{equation}
d_{C}:=d_{PR}(t=t_0)=a(t_0)\mathfrak{f}(r)=\frac{a(t_0)}{a(t)}d_{PR}
\end{equation}

\subsection{Redshift}

Siano~\(\lambda_e\) la lunghezza d'onda di un segnale emesso
localmente da una sorgente e~\(\lambda_{ob}\) quella ricevuta
dall'osservatore si definisce~\emph{redshift} la quantità:
\begin{equation}
z=\frac{\lambda_{ob}-\lambda_e}{\lambda_e}=\frac{\lambda_{ob}}{\lambda_e}-1
\end{equation}

Si cosideri l'emissione consecutiva di due fotoni che avviene rispettivamente a~~\(t_e\)
e~\(t_e+\delta t_e\). Seguendo la geodetica dei fotoni
ha~\(\mathrm{d}s^2=0\) e si può assumere~\(\mathrm{d}\theta=\mathrm{d}\varphi=0\) senza perdere
di generalità. Dall'equazione della metrica (\ref{eq1:frw}) si ottiene:
\begin{equation}
\int_{t_{e}}^{t_{ob}}\frac{c\,\mathrm{d}t}{a(t)}=\int_{0}^{r}\frac{\mathrm{d}r'}{\sqrt{1-kr'}}=\mathfrak{f}(r)\equiv \int_{t_{e}+\delta t_e}^{t_{ob}+\delta t_{ob}}\frac{c\,\mathrm{d}t}{a(t)}
\end{equation}

dal momento che~\(\mathfrak{f}(r)\) dipende solo dalla geometria.
Si possono considerare \(\delta t_{e}\) e~\(\delta t_{ob}\) piccoli, in modo che \(a=cost\) nel tempo di integrazione. Per convenzione, le quantità che si riferiscono al tempo di emissione vengono scritte senza pedici, mentre quelle che si riferiscono al tempo di osservazione (\emph{oggi}) assumono pedice $0$. Si ottengono le seguenti relazioni:
\begin{equation}
\frac{\delta t_0}{a_0}=\frac{\delta t}{a};\qquad a\nu=a_0\nu_0;\qquad 1+z=\frac{a_0}{a}
\end{equation}

Possiamo quindi pensare~\(z\) come la misura di quanto è
variato il fattore di scala dal tempo dell'emissione del fotone. Un
risultato analogo poteva essere ottenuto ipotizzando osservatore e
sorgente in moto relativo (effetto Doppler), ma non esiste una velocità
effettiva di allontanamento: è lo spazio che si dilata.


\subsection{Distanza di luminosità}

Si definisce a partire dalla relazione tra luminosità~\(L [erg/s]\)
e flusso~\(l\) considerando i seguenti effetti:

\begin{itemize}
\item
  \emph{Dilatazione temporale}: $\d{t} / \d{t_0} = a / a_0$; 
\item
  \emph{Redshift (cambiamento di energia)}: $\d{E}_0 / \d{E} = a / a_0$
\item
  \emph{Variazione della geodetica}: $d(t_0) = a_0\: r$.
\end{itemize}
\begin{equation}
l=\frac{L}{4\pi d_L^2}\equiv\frac{L}{4\pi a_0^2r^2}\frac{a^2}{a_0^2}=\frac{L}{4\pi a_0^2r^2}\frac{1}{(1+z)^2}\qquad d_L := a_0 r (1+z)
\end{equation}

\subsection{Distanza (di diametro) angolare}

Si definisce misurando le dimensioni angolari di un righello standard,
ossia di una classe di oggetti astrofisici che hanno la stessa
dimensione intrinseca~\(D\) posti a distanza diversa (nella
realtà non esistono,~\emph{forse...}). Si parte dall'equazione (\ref{eq1:frw}) e si
assume~\(\mathrm{d}r=\mathrm{d}\varphi = 0\) e~\(\mathrm{d}t=0\) (poichè si osservano le
due estremità dell'oggetto contemporaneamente):
\begin{equation}
(D^2=)\, \mathrm{d}s^2=a^2r^2\mathrm{d}\theta^2\rightarrow \frac{D}{\mathrm{d}\theta}=ar\qquad d_A:=a(t)\, r
\end{equation}

Confrontandola con la (12) si ottiene la~\textbf{relazione di dualità}:
\begin{equation}
d_A=\frac{d_L}{(1+z)^2}
\end{equation}

Queste due distanze sono definite diversamente, non ce ne è una ``più
fisica'' dell'altra: la scelta va fatta in base al tipo di osservazione.
Se si utilizza la luminosità intrinseca per
stimare la distanza (e.g. Cefeidi, SNe, Tip RGB, \ldots{}), si applica la~\(\mathrm{d}_L\); se
esistesse un righello standard cosmologico (e.g. orizzonte cosmologico),
si applica la definizione~\(\mathrm{d}_A\). Questa relazione DEVE
valer se l'universo fosse Robertson-Walker, mentre in un universo
statico si avrebbe~\(d_A=d_L\).~

Esistono altre definizioni di distanza: parallasse, moti propri e così
via\ldots{} Tutte tendono allo stesso valore per \(z\)
piccoli.

\par\null
\begin{definition}
  \textbf{Sommario:}
\begin{align*}
  d_{PR} & =a\,\mathfrak{f}(r)  \\ 
  d_{C} & =\frac{a_0}{a}\,\mathrm{d}_{PR}  \\ 
  d_L & = a_0 \, r\, (1+z)  \\ 
  d_A & =a\, r  \\
  1+z & =\frac{a_0}{a}
  \end{align*}
\end{definition}

\subsection{Legame con gli osservabili}

Tutte le distanze possono essere riscritte in funzione di un unico
parametro $a=a(t)$. Sviluppandolo in serie per $t\to t_0$ si ha:
\begin{equation}
a(t)\simeq a_0\left ( 1+H_0 (t-t_0)-\frac{1}{2}q_0\,H_0^2(t-t_0)^2  \right )\qquad 
H:=\frac{\dt{a}}{a}; \quad
q:=-\frac{\ddt{a} a}{\dt{a}^2} 
\end{equation}

dove $q$ è detto \emph{parametro di decelerazione} ed è adimensionale. Viene definito col meno perché in passato i modelli facevano propendere per un'espansione dell'universo decelerata. 

\subsubsection{Hubble test applicato a $d_L$}
Applicando lo sviluppo alla \emph{distanza di luminosità} si ha:
\begin{equation}
d_L\simeq \frac{cz}{H_0}\left ( 1 + \frac{1}{2}(1-q_0)z  \right )
\end{equation}

da cui, applicando il logaritmo e sviluppando per \(\log\left(1+x\right)\rightarrow\ \frac{x}{\ln10}\), si
ottiene la relazione:
\begin{equation}
\log d_L = \log cz-\log H_0+0.217(1-q_0)z
\end{equation}

oppure, esplicitandola in magnitudini ($ m-M=\mu=5\log\frac{d}{\mathrm{10\,pc}} $):
\begin{equation}
\underline{m} - \underline{\underline{M}} = 25 + 5\log c\underline{z}-5\log H_0 + 1.086~(1-q_0)~\underline{z}
\end{equation}

dalla quale tramite $\textrm{\underline{osservazioni}}$ e $\underline{\underline{\textrm{modelli}}} $ (e.g. $M_V=-19$ per le SN-TN) è possibile calibrare $H_0$ (qui in km/s/Mpc) localmente e $q_0$ a $z$ poco più elevato.

\subsubsection{Hubble test applicato alla densità di oggetti}
In un universo euclideo si ha che la densità di oggetti $N\propto d_L^3$. Ipotizzando che tutti gli oggetti abbiano la stessa luminosità, quelli con flusso superiore ad $l$ saranno contenuti in un cubo di densità $N(>l)\propto l^{-\frac{3}{2}}$. Quindi si dovrà osservare la relazione: 
\begin{equation}
\log N(>l) \propto -1.5 \log l \propto 0.6~m
\end{equation}

Rilassando il vincolo di universo euclideo è necessario calcolare l'integrale:
\begin{equation}
N(>l)= 4\pi \int_{0}^{r}\frac{r'^2}{\sqrt{1-kr'^2}}~n(t(r'))~a(t(r'))^3 \mathrm{d}r'
\end{equation}
che si risolve assumendo che esista una classe di oggetti celesti per la quale la densità rimane costante nel tempo ($n_0a_0^3=na^3$). In passato le prime stime furono fatte mediante le galassie, oggi sappiamo che non esistono oggetti che si comportano così per tempi cosmologici.
In questa assunzione si può espandere la funzione integranda $\frac{r'^2}{\sqrt{1-kr'^2}}$ per $r\rightarrow 0$ per ottenere la relazione:
\begin{equation}
\log N(>l) = cost + 3\log z - \frac{3}{2\ln 10}~(1+q_0)~z
\end{equation}
in cui $H_0$ è in pancia alla $cost$. Si noti che il numero di oggetti non diverge (cfr. Paradosso di Olbers), ci arriva informazione solo da una parte di Universo, quello in connessione causale con noi.


